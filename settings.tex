\documentclass[onecolumn,11pt,a4paper]{article}
% This first part of the file is called the PREAMBLE. It includes
% customizations and command definitions. The preamble is everything
% between \documentclass and \begin{document}.

% -------------------------------------------------------------------- %
% Debug mode/ clean mode:
% -------------------------------------------------------------------- %
%%%%%%%%%%%%%%%%%%%%%%%%%%%%%%%%%%%%%%%%%%%%%%%%%%%%%%%%%%%%%%%
\newif\ifdebug
%
\debugtrue
%\debugfalse
%
%%%%%%%%%%%%%%%%%%%%%%%%%%%%%%%%%%%%%%%%%%%%%%%%%%%%%%%%%%%%%%%


\usepackage[utf8]{inputenc}        % correct understanding of the input
\usepackage{lmodern}               % using modern encoding for the output
\usepackage[T1]{fontenc}           % and correct encoding of the out font
\usepackage[english]{babel}

%\usepackage{todonotes}
\ifdebug
\usepackage[top=1.35in, bottom=1.5in, left=1in, right=1in, headheight=14.5pt, showframe]{geometry}  % set the margins

\else
\usepackage[top=1.35in, bottom=1.5in, left=1in, right=1in, headheight=14.5pt]{geometry}
\fi
\usepackage{latexsym}              % symbols
\usepackage{amsmath}               % great math stuff
\usepackage{amssymb}               % great math symbols
\usepackage{amsfonts}              % for blackboard bold, etc
\usepackage{amsthm}                % for theorems, 

\usepackage{amsbsy}       % other symbols
% Experimental command, from http://tex.stackexchange.com/a/51883/97964
\allowdisplaybreaks
% bbm package for \mathbbm{1} for indicator functions, http://tex.stackexchange.com/a/26640/97964
\usepackage{bbm}

% Use TikZ figures. They are written in external files tikz_figure_...tex
\usepackage{tikz}                  % to make drawing with TikZ
\usetikzlibrary{arrows,positioning,shapes}
% Enf ot TikZ libraries
\usepackage{tikzsymbols}

\usepackage{enumerate}             % for the \enumerate command
\usepackage{enumitem}
\setlist{nolistsep}

\usepackage{xparse}                % more robust \newcommand
\usepackage{hyperref}              % hyperlinks
\usepackage{url}

\usepackage{color, xcolor}                 % use colors with, e.g., \textcolor{red}{text}
\definecolor{darkgray}{rgb}{0.15,0.15,0.15}   % Define a new color rgb(38, 38, 38)
\definecolor{lightgray}{rgb}{0.94,0.94,0.94}  % Define a new color rgb(239, 239, 239)
\definecolor{lightlightgray}{rgb}{0.97,0.97,0.97}  % Define a new color rgb(247, 247, 247)
\definecolor{darkred}{rgb}{0.80,0.00,0.00}    % Define a new color rgb(204, 0, 0)
\definecolor{darkgreen}{rgb}{0.00,0.70,0.00}    % Define a new color rgb(0,178,0)
\definecolor{darkblue}{rgb}{0.00,0.00,0.70}    % Define a new color rgb(0,0,178)

\ifdebug
\pagecolor{gray}
\usepackage[left]{showlabels}
\renewcommand{\showlabelfont}{\ttfamily\tiny}
\fi

% remove pagebreak after chapter

%

% font and math style
\usepackage[p,osf]{cochineal}
\usepackage[scale=.95,type1]{cabin}
\usepackage[cochineal,bigdelims,cmintegrals,vvarbb]{newtxmath}
\usepackage[zerostyle=c,scaled=.94]{newtxtt}
\usepackage[cal=boondoxo]{mathalfa}
%

\usepackage{graphicx}              % to include figures
\graphicspath{{fig/}}              % directory in which figures are stored
\usepackage{caption}               % add a caption to figures
\usepackage{subcaption}            % and allow multi figures
\usepackage{multicol}              % multi column environment
\usepackage{float}                 % allow more control to float and figures positions
\usepackage{framed}                % \framed command to frame an important part
\usepackage[framemethod=tikz]{mdframed}
%\usepackage{mathpazo}               % charter, fourier, mathpazo, times
\usepackage{enumitem}              % For \begin{enumerate}, switch back to itemize if it fails
\usepackage{glossaries}            % To create glossaries

% Custom package for the better footnote symbols, http://ctan.org/pkg/footmisc
\usepackage[bottom]{footmisc}
% \usepackage[perpage,para,symbol*]{footmisc}

% Tips from http://tex.stackexchange.com/a/205133/97964
%\usepackage[nameinlink]{cleveref}


% Various theorems, numbered by section (required amsthm)
\numberwithin{equation}{section}
%
\theoremstyle{plain}  % Plain style for theorem, defn, lemma, proposition, corollary
\newtheorem{thm}{Theorem}[section]
\newtheorem{lem}[thm]{Lemma}
\newtheorem{prop}[thm]{Proposition}
\newtheorem{cor}[thm]{Corollary}

\theoremstyle{definition}
\newtheorem{defn}[thm]{Definition}
\newtheorem{notation}[thm]{Notation}
\newtheorem{problem}{Problem}[section]
\newtheorem{question}{Question}[section]

\theoremstyle{remark}  % Remark style for remark, example, examples
\newtheorem{rmk}[thm]{Remark}
\newtheorem{ex}[thm]{Example}

% Custom theorem and lemma
\theoremstyle{plain}
\newtheorem{innercustomgeneric}{\customgenericname}
\providecommand{\customgenericname}{}
\newcommand{\newcustomtheorem}[2]{%
  \newenvironment{#1}[1]
  {%
   \renewcommand\customgenericname{#2}%
   \renewcommand\theinnercustomgeneric{##1}%
   \innercustomgeneric
  }
  {\endinnercustomgeneric}
}

\newcustomtheorem{customthm}{Theorem}
\newcustomtheorem{customlem}{Lemma}
\newcustomtheorem{customcor}{Corollary}


\usepackage{lastpage,fancyhdr}     % customize the headers and footers
\pagestyle{fancy}
\fancyhf{} %delete everything
\renewcommand{\headrulewidth}{0.5pt} % 0.5pt
\renewcommand{\footrulewidth}{0.3pt} % 0.3pt
\fancyhead[R]{\small\nouppercase\leftmark} %section title
\fancyhead[L]{\thepage} %page number on all pages


%% Custom commands
% horizontal lines, as a <hr> tag in HTML
\providecommand*{\hr}[1][class-arg]{%
    \hspace*{\fill}\hrulefill\hspace*{\fill}
    \vskip 0.65\baselineskip
}

\newcommand{\todo}[1]{\ifdebug \textcolor{red}{\small (#1) \marginpar{\small\textcolor{red}{to-do}}} \fi}

\newcommand{\dnote}[1]{\ifdebug \textcolor{blue}{\small (David: #1)} \fi}


\newcommand{\qq}[1]{``#1''}
\newcommand{\q}[1]{`#1'}

\newcommand{\norm}[1]{\left\lVert#1\right\rVert}
\newcommand{\brac}[1]{\left(#1\right)}
\newcommand{\sbrac}[1]{\left[#1\right]}
\newcommand{\cbrac}[1]{\left\{#1\right\}}
\newcommand{\vbrac}[1]{\left\langle#1\right\rangle}
\newcommand{\eval}[3]{\left.#1\right|_{#2}^{#3}}
\newcommand{\set}[1]{\left\{ #1 \right\}}
\newcommand{\abs}[1]{\left| #1 \right|}

\NewDocumentCommand{\prob}{mg}{
    \IfValueTF{#2}{
        P(#1 \,\vert\, #2)
    }{
        P(#1)
    }
}

%
\newcommand\restr[2]{{
  \left.\kern-\nulldelimiterspace % automatically resize the bar with \right
  #1 % the function
  %\vphantom{\big|} % pretend it's a little taller at normal size (comment out if not needed)
  \right|_{#2} % this is the delimiter
  }}

% fix chi subscript
\let\latexchi\chi
\makeatletter
\renewcommand\chi{\@ifnextchar_\sub@chi\latexchi}
\newcommand{\sub@chi}[2]{% #1 is _, #2 is the subscript
  \@ifnextchar^{\subsup@chi{#2}}{\latexchi^{}_{#2}}%
}
\newcommand{\subsup@chi}[3]{% #1 is the subscript, #2 is ^, #3 is the superscript
  \latexchi_{#1}^{#3}%
}
\makeatother

%% Custom mathematical symbols
% Short symbols
\newcommand{\e}{\mathrm{e}}
\renewcommand{\d}{\,\mathrm{d}}
% Sets
\newcommand{\N}{\mathbb{N}}
\newcommand{\Z}{\mathbb{Z}}
\newcommand{\R}{\mathbb{R}}
\newcommand{\C}{\mathbb{C}}

% = Symbols
\newcommand{\eqdef}{\mathop{\mathrel{\stackrel{\smash{\scriptscriptstyle\mathrm{def}}}{=}}}}
% Functions
\newcommand{\bra}[1]{\langle#1|}
\newcommand{\ket}[1]{|#1\rangle}
\newcommand{\braket}[2]{\langle#1|#2\rangle}
\newcommand{\corre}[1]{\langle#1\rangle}
\newcommand{\inprod}[2]{\left\langle#1,#2\right\rangle}
\newcommand{\presen}[2]{\langle#1:#2 \rangle}
\renewcommand{\arg}{\mathop{\mathrm{arg}}}
\newcommand{\conj}[1]{\overline{#1}}
\newcommand{\convol}{\mathop{\ast}}
\newcommand{\sign}{\mathop{\mathrm{sign}}}
\newcommand{\sinc}{\mathop{\mathrm{sinc}}}
\newcommand{\Real}{\mathfrak{R}}
\newcommand{\Imag}{\mathfrak{I}}
\newcommand{\dom}{\mathrm{Dom}}
\newcommand{\Tr}[1]{\mathrm{Tr}\left(#1\right)}
\newcommand{\Span}{\mathrm{Span}}
\newcommand{\Id}{\mathrm{Id}}
\newcommand{\F}{\mathcal{F}}
\newcommand{\cL}{\mathcal{L}}